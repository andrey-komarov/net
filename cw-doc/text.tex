
%%% Local Variables: 
%%% mode: latex
%%% TeX-master: t
%%% End: 

\section{Обзор работы}

\subsection{Наименование}

Многопользовательское сетевое приложение <<Радио>>.

\subsection{Описание поведения клиента}

При запуске пользователь вводит адрес сервера, осуществляющего вещание
песен. После этого, клиент запрашивает у сервера список имеющихся у
него песен и выводит пользователю их список. Пользователь выбирает,
какую из них прослушать, клиент посылает серверу запрос на
прослушивание этой песни. Песня запускается.

\subsection{Описание поведения сервера}

Сервер ждёт подключения клиентов. Для каждого подключившегося клиента
обрабатываются два типа запросов:

\begin{itemize}
\item Получить список песен: отдать подключившемуся пользователю
  список имеющихся песен.
\item Проиграть песню: отдать пользователю запрошенную песню. 
\end{itemize}

\section{Описание протокола}

Сервер слушает TCP-запросы, приходящие на порт 3333.

\subsection{Процедура авторизации}

Каждый раз, когда к серверу кто-то подключается, подключившийся,
прежде, чем посылать свой запрос, должен пройти процедуру
аутентификации. 

У каждого клиента есть пара из публичного и приватного 1024-битных
ключей. 

\begin{figure}
  \label{aut}
  \centering

  \begin{tabular}{|r|l|l|l|}
    \hline
    байт & Клиент $\to$ сервер & Сервер $\to$ клиент & байт \\
    \hline
    128 & $key_{pub}$ & случайных байты $v$ & 128 \\
    \hline
    256 & $\operatorname{sign}(v, key_{priv})$ & &\\
    \hline
  \end{tabular}

  \caption{Процедура авторизации}
\end{figure}

Подпись и верификация подписи производятся алгоритмом Эль-Гамаля.

Если после проверки сервером подписи не произошло ошибки, авторизация
считается пройденной. Дальнейшие запросы обрабатываются через этот же
сокет.

На рисунке~\ref{aut} изображена схема работы.

\subsection{Запрос списка песен}

На рисунке~\ref{songs} изображена схема работы.

Описание песни состоит из:

\begin{enumerate}
\item 32 байта~--- уникальный идентификатор песни;
\item 4 байта~--- длина отображаемого названия $l$;
\item $l$ байт~--- отображаемое название в кодировке UTF-8.
\end{enumerate}

\begin{figure}
  \label{songs}
  \centering

  \begin{tabular}{|r|l|l|l|}
    \hline
    байт & Клиент $\to$ сервер & Сервер $\to$ клиент & байт \\
    \hline
    1 & \texttt{0x01} & & \\
    \hline
    & & Число песен $n$ & 4 \\
    \hline
    & & $n$ описаний песен & \\
    \hline
  \end{tabular}

  \caption{Запрос списка песен}
\end{figure}

\subsection{Запрос песни}

На рисунке~\ref{song} изображена схема работы.

\begin{itemize}
\item \texttt{0x00} передаётся, если песня с таким идентификатором
  недоступна. В этом случае, больше ничего не передаётся.
\item \texttt{0x01} передаётся, если песня найдена. В этом случае,
  общение продолжается.
\end{itemize}

\begin{figure}
  \label{song}
  \centering

  \begin{tabular}{|r|l|l|l|}
    \hline
    байт & Клиент $\to$ сервер & Сервер $\to$ клиент & байт \\
    \hline
    1 & \texttt{0x02} & & \\
    \hline
    32 & уникальный идентификатор песни & & \\
    \hline
    & & \texttt{0x00} или \texttt{0x01} & 1 \\
    \hline
    & & длина \texttt{mp3}-файла $n$ & 4 \\
    \hline
    & & \texttt{mp3}-файл & $n$ \\
    \hline
  \end{tabular}

  \caption{Запрос списка песен}
\end{figure}
